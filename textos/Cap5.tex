\section{Análise de Logs}
    
    \vspace{10.5cm}
    
    \hspace{1cm}
    Aplicações distintas são projetadas para resolver determinado tipo de problema no cotidiano. Uma vez interligadas pela \textit{Internet}, informações heterogêneas são produzidas e trocadas. Para manter o devido controle do que as aplicações estão gerando em tempo de execução, desenvolvedores mantêm diferentes tipos de \textit{Logs}. Log é o nome dado a um arquivo devidamente formatado, para representar informações de: eventos ocorridos em um sistema computacional, segurança de um dispositivo, erros ocorridos na utilização de uma aplicação, instalação de programas e registros auxiliadores na depuração de \textit{softwares}.
    
    \vspace{4mm}
    
    \hspace{1cm}
    De acordo com \citeonline{kent2006}, logs podem ser classificados em cinco categorias, como se seguem:
    
    \begin{itemize}
        \item \textbf{Eventos}: tipo de registro geralmente mantido pelo sistema operacional hospedeiro, para armazenar informações como: lista do que foi executado, a data em que cada ação ocorreu e o resultado de cada uma;
        \item \textbf{Segurança}: arquivos formatados que contêm dados auditáveis como: tentativas de \textit{logon} (bem sucedidas ou não), mudanças de políticas de segurança, acesso a arquivos e execução de processos;
        \item \textbf{Erros}: algumas aplicações armazenam dados sobre erros ocorridos em tempo de execução, para ser possível rastrear a origem de um problema decorrente de ataques ou não;
        \item \textbf{Instalação}: aplicativos podem registrar, de forma persistente, o que ocorreu no momento inicial de sua instalação, bem como aquilo que aconteceu após a integração de uma atualização;
        \item \textbf{Depuração}: alguns desenvolvedores de \textit{software} disponibilizam executáveis em modo de depuração, seja para que os usuários testem e reportem algum código de erro, seja para resolver problemas operacionais específicos.
    \end{itemize}
    
    \subsection{Exercícios}

    \begin{example}[\quad \large Análise de Logs] \label{cap5_exercicios}
        \begin{enumerate}
            \item Explique, com suas palavras, para que serve cada tipo de log apresentado neste capítulo.
            \item Supondo que uma intrusão tenha ocorrido em um sistema, qual(is) tipo(s) de log o perito pode consultar para tentar obter alguma pista sobre o ocorrido ?
            \item Pesquise no \textit{Google} como acessar logs de eventos ocorridos no sistema operacional que você usa no cotidiano.
        \end{enumerate}
    \end{example}
    
\newpage