\section{Análise de Inteligência}
    
    \vspace{10.5cm}
    
    \hspace{1cm}
    Em uma investigação, pode ser necessário pesquisar mais informações sobre determinado alvo, seja ele um indivíduo, uma empresa ou até um governo. Diante disso, existem bases de dados públicas, tanto nacionais como estrangeiras, para consultar informações específicas. Inclusive, há \textit{softwares} agregadores de informações de fontes abertas, como a ferramenta \textit{Maltego}.
    
    \vspace{4mm}
    
    \hspace{1cm}
    Em consonância com \citeonline{barreto2020}, entende-se como fonte aberta aquela disponível na \textit{Internet}, que não impõe restrição de acesso às suas informações. Por outro lado, existem fontes fechadas que não permitem a nenhum usuário não autorizado acessá-las. Seus dados são protegidos, ou negados. Quando negados, é preciso obter um mandado de busca para torná-los acessíveis \cite{barreto2020}. Portanto, trataremos das fontes abertas nacionais em nossa revisão.
        
    \vspace{4mm}
    
    \hspace{1cm}
    Em nosso dia a dia, é comum utilizarmos um motor de busca como o \textit{Google}, para tentar encontrar  informações, mas, no contexto de uma investigação, deseja-se obter informações específicas sobre um alvo. O site provido pela \textit{Google} tem mecanismos de busca avançada. Os chamados \textit{Google Dorks} conseguem refinar consultas, reduzir o escopo da pesquisa e, então, o número de resultados. Por meio da utilização de operadores lógicos, como "$|$" (OU), "$+$" (E) e "$-$" (negação), é possível formatar cadeias de busca. Logo, se quisermos procurar um termo A \textit{e} um termo B, basta usar o operador "$+$" para conjugar ambos.
        
    \vspace{4mm}
    
    \hspace{1cm}
    De maneira geral, pode-se usar redes sociais como \textit{Facebook}, \textit{Instagram} e \textit{Twitter}, para pesquisar informações pessoais sobre algum indivíduo. Além disso, é possível localizar dados de titularidade de um determinado domínio da \textit{Internet}, usando a ferramenta \textit{Whois} (como apresentado na seção \ref{cap4_ferramentas}). Por fim, com dados suficientes em mãos, o perito pode organizar uma linha do tempo que interliga indivíduos, grupos, localizações e sites, na ferramenta Maltego \cite{bazzell2022}.
    
    \subsection{Exercícios}
    
    \begin{example}[\quad \large Análise de Inteligência] \label{cap6_exercicios}
        \begin{enumerate}
            \item Diferencie fontes abertas de fontes fechadas de informação.
            \item Escolha um termo de pesquisa para consultar no \textit{Google}. Em seguida, experimente realizar a mesma busca, usando os operadores "$|$" (OU), "$+$" (E) e "$-$" (negação). Quando comparado com o da primeira, o número de resultados da segunda busca diminuiu ?
            \item Assumindo que um perito tenha em mãos informações encontradas em fontes abertas, qual ferramenta ele pode usar, para agregá-las e facilitar sua análise ?
        \end{enumerate}
    \end{example}

\newpage