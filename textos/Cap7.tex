\section{Análise de Malware}

    \vspace{10.5cm}

    \hspace{1cm}
    Tendo em vista o que são códigos maliciosos e formas de ataques (ver subseção \ref{cap1_visao_geral_malware}), a análise de \textit{malware} tem por objetivo garantir que o profissional saiba exatamente o que ocorreu com máquinas infectadas por aplicações maliciosas. Mais especificamente, a intenção do analista é descobrir como um executável se comporta no computador da vítima \cite{sikorski2012}.
    
    \vspace{4mm}
    
    \hspace{1cm}
    Para a execução eficiente desses artifícios, o profissional precisa ter conhecimento aplicado em ciência da computação. Porquanto, para conseguir compreender a maneira conforme um programa é executado num computador, o experto deve utilizar da análise crítica desenvolvida no estudo de disciplinas como Sistemas Operacionais, Arquitetura e Organização de Computadores, Redes, entre outras. Ainda assim, existem diversas ferramentas que auxiliam o técnico a completar o exame de \textit{malwares}. Afinal, os diferentes níveis de abstrações, oferecidos pelos utensílios, podem acelerar este processo na totalidade.
    
    \subsection{Técnicas}
    
    \hspace{1cm}
    O analista de \textit{malware} dispõe de três técnicas para tentar classificar programas maliciosos conforme os comportamentos analisados. A primeira técnica chama-se Análise Dinâmica. Ela é usada quando o profissional deseja entender o que acontece ao passo que um dado programa nocivo executa. O segundo esquema é denominado Análise Estática. Nesse caso, o analista utiliza ferramentas específicas para obter informações comportamentais de um programa, sem executá-lo. Uma vez com esses dados em mãos, o perito pode usufruir deles, numa possível heurística dinâmica. A última artimanha é conhecida como Análise Híbrida, cujo diferencial é juntar a Análise Estática com a Dinâmica \cite{monteiro2018}.
        
    \subsection{Ferramentas}
    
    \hspace{1cm}
    Em concordância com \citeonline{monteiro2018}, categorizaremos algumas ferramentas que usualmente complementam a reprodução das técnicas dispostas. Na análise estática, é comum operar programas como \textit{OllyDBG}, \textit{IDA Pro}, \textit{WinDBG} e \textit{x64dbg}. Todos eles retornam valores em linguagem de máquina que podem ser utilizados em uma engenharia reversa no futuro. Para se ter uma prévia do que o trapaceiro busca atingir na máquina da vítima, é usado um virtualizador de máquinas como \textit{VMWare} ou \textit{Oracle VirtualBox}, para executar uma amostra. Caso o examinador deseje usar ambas as técnicas, é possível manusear a suíte \textit{SysInternals} em conjunto com as demais equipagens para obter resultados satisfatórios.
    
    \vspace{4mm}

    \hspace{1cm}
    Por outro lado, existem plataformas na \textit{Web} que podem auxiliar peritos menos experientes nessa subárea, como nos casos das ferramentas \textit{Cuckoo Sandbox} e \textit{Falcon Sandbox}. Ambas são aplicações transparentes que permitem ao usuário o envio de uma amostra de \textit{malware} para análise automatizada. Esta, então, é executada em ambiente virtual e diversas informações são coletadas em tempo de execução, para que o perito possa analisar seu comportamento.

    \subsection{Exercícios}
    
    \begin{example}[\quad \large Análise de Malware] \label{cap7_exercicios}
        \begin{enumerate}
           \item Para que serve a Análise Dinâmica ?
           \item Para que serve a Análise Estática ?
           \item Qual ferramenta, dentre as apresentadas neste capítulo, um analista de malware deve usar para ver quais instruções são executadas em dado momento por um programa ?
           \item Escolha um arquivo executável e submeta para a ferramenta Falcon Sandbox (acesse \url{https://www.hybrid-analysis.com/}). Relate quais informações a ferramenta conseguiu produzir ao fim de sua análise automatizada.
        \end{enumerate}
    \end{example}

\newpage