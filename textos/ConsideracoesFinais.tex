\section{Considerações Finais}
    
    \vspace{10.5cm}

    \hspace{1cm}
    Enfim, espera-se que o leitor tenha compreendido os fundamentos teóricos para estudar Forense Digital e Resposta a Incidentes. Vale ressaltar que essa área é multi-disciplinar, e o estudante não precisa aprofundar-se em todas as subáreas apresentadas. O mais comum é que o futuro especialista foque em pelo menos três campos, e compreenda quando e o que usar nos outros \cite{roberts2016}. Outrossim, há desafios inerentes ao tratamento de grandes volumes de dados, e, pesquisas recentes propõem formas de realizar Análise Forense com emprego de Inteligência Artificial, e em dispositivos móveis.

    \vspace{4mm}

    \hspace{1cm}
    Em suma, e em consonância com \citeonline{garfinkel2010}, os maiores desafios da área de FDRI são: dificuldade para realizar cópias de discos cada vez maiores; utilização frequente de memórias \textit{flash}; uso de diferentes sistemas operacionais e sistemas de arquivos; dificuldade para analisar múltiplos dispositivos em casos complexos; existência de dados criptografados que, quando recuperados, não são facilmente processados; utilização de computação na nuvem; existência de vírus que residem na memória principal do computador e exigem análise forense mais específica.

    \vspace{4mm}

    \hspace{1cm}
    Pesquisadores começaram a aplicar conhecimentos de Inteligência Artificial (IA) nas diversas fases do processo de análise forense clássico, e também a explorar análise forense em dispositivos móveis. Como evidenciado por \citeonline{du2020}, IA vem sendo utilizada nas fases de Aquisição de cópias forenses, na etapa de Exame forense, em Análise forense e Apresentação de evidências, assim como diversos algoritmos são usados para: recuperação de dados, triagem forense de dispositivos, análise de redes, análise forense de dados criptografados (criptoanálise), reconstrução de eventos e linha do tempo, análise forense em aparelhos multimídia e \textit{fingerprinting}. Além disso, \citeonline{sharmalc2020} propuseram um novo procedimento forense, para coletar e configurar evidências em dispositivos Android; a abordagem é chamada \textit{Enhanced Mobile Cloud Forensic Process} e aproveita os registros e arquivos de aplicativos de redes sociais, para montar o perfil de um suspeito e rastrear diversas informações no celular e nos servidores na nuvem ligados às suas contas virtuais.

\newpage